\documentclass[
12pt, % slightly smaller to fit everything
portrait,
margin=5mm,
innermargin=5mm,
blockverticalspace=5mm, % tighter vertical spacing
colspace=10mm,
subcolspace=10mm
]{tikzposter}
\geometry{paperwidth=22in,paperheight=28in}
\makeatletter
\setlength{\TP@visibletextwidth}{\textwidth - 2\TP@innermargin}
\setlength{\TP@visibletextheight}{\textheight - 2\TP@innermargin}
\makeatother

% === PACKAGES ===
\usepackage[utf8]{inputenc}
\usepackage{amsmath,amssymb,amsfonts,amsthm}
\usepackage{graphicx}
\usepackage{enumitem}
\usepackage{lmodern}
\usepackage{microtype}
\usepackage{adjustbox}
\renewcommand{\familydefault}{\sfdefault}
\usepackage{sfmath}

% === BERKELEY COLORS ===
\definecolor{BerkeleyBlue}{HTML}{003262}
\definecolor{CaliforniaGold}{HTML}{FDB515}
\definecolor{FoundersRock}{HTML}{3B7EA1}
\definecolor{LightGray}{HTML}{F5F5F5}

% === CUSTOM THEME ===
\definecolorstyle{BerkeleyTheme}{
	\definecolor{colorOne}{named}{BerkeleyBlue}
	\definecolor{colorTwo}{named}{LightGray}
	\definecolor{colorThree}{named}{CaliforniaGold}
}{
	\colorlet{backgroundcolor}{white}
	\colorlet{framecolor}{BerkeleyBlue}
	\colorlet{titlefgcolor}{white}
	\colorlet{titlebgcolor}{BerkeleyBlue}
	\colorlet{blocktitlebgcolor}{BerkeleyBlue}
	\colorlet{blocktitlefgcolor}{white}
	\colorlet{blockbodybgcolor}{white}
	\colorlet{blockbodyfgcolor}{black}
}

\defineblockstyle{BerkeleyBlock}{
	titlewidthscale=1, bodywidthscale=1, titlecenter,
	titleoffsetx=0pt, titleoffsety=0pt, bodyoffsetx=0pt, bodyoffsety=0pt,
	bodyverticalshift=0pt, roundedcorners=0, linewidth=2pt,
	titleinnersep=10pt, bodyinnersep=14pt % tighter padding
}{
	\begin{scope}[line width=\blocklinewidth, rounded corners=\blockroundedcorners]
		\ifBlockHasTitle
		\draw[color=framecolor, fill=blocktitlebgcolor]
		(blockbody.south west) rectangle (blocktitle.north east);
		\draw[color=framecolor, fill=blockbodybgcolor]
		(blockbody.south west) rectangle (blockbody.north east);
		\else
		\draw[color=framecolor, fill=blockbodybgcolor]
		(blockbody.south west) rectangle (blockbody.north east);
		\fi
	\end{scope}
}

\definelayouttheme{BerkeleyLayout}{
	\usecolorstyle{BerkeleyTheme}
	\usebackgroundstyle{Default}
	\usetitlestyle{Filled}
	\useblockstyle{BerkeleyBlock}
}
\usetheme{BerkeleyLayout}

% === TITLE ===
\title{\parbox{0.95\linewidth}{\centering\huge\textbf{Predicting Benign Overfitting via Spectral Geometry}}}
\author{\Large Divit Rawal}
\institute{\large UC Berkeley}

% ==============================================================================
\begin{document}
	\maketitle
	
	% ==============================================================================
	% TOP SECTION: THEORY & MATH
	% ==============================================================================
	\begin{columns}
		% ---------------- LEFT COLUMN ----------------
		\column{0.5}
		
		% === MOTIVATION ===
		\block{Motivation}{
			\textbf{Classical intuition:} More parameters $\Rightarrow$ more overfitting.
			
			\textbf{Modern ML:}
			\begin{itemize}[leftmargin=*, itemsep=0.25em]
				\item Deep / overparameterized models ($p \gg N$) often interpolate and still generalize.
				\item Some interpolating solutions are \emph{benign}, others \emph{catastrophic}.
				\item Width or sample size alone do \underline{not} predict which.
			\end{itemize}
			
			\textbf{Question.} Can we predict benign vs non-benign interpolation
			using only the unlabeled feature matrix $X$?
			
			\textbf{Thesis.} Generalization is controlled by a simple spectral quantity
			of the feature covariance, not by $p$ or $N$ in isolation.
		}
		
		% === MODEL + CERTIFICATE INTUITION ===
		\block{Setup and Certificate}{
			\small
			\textbf{Model.} Fixed features $\phi(x)\in\mathbb{R}^p$ and realizable linear regression
			\[
			y = \phi(x)^\top \theta^\star + \varepsilon,\quad
			\mathbb{E}[\varepsilon\mid x]=0,\;
			\mathbb{E}[\varepsilon^2\mid x]\le\sigma^2,\;
			\|\theta^\star\|_2\le B.
			\]
			With $N$ samples, $X\in\mathbb{R}^{N\times p}$, empirical covariance
			$\hat\Sigma = \tfrac1N X^\top X$, population $\Sigma = \mathbb{E}[\phi(x)\phi(x)^\top]$.
			
			\vspace{0.3em}
			\textbf{Ridge predictor.}
			\[
			\hat\theta_\lambda
			= (\hat\Sigma + \lambda I)^{-1} \tfrac1N X^\top y.
			\]
			
			\vspace{0.3em}
			\textbf{Effective dimension.}
			\[
			d_\lambda(\hat\Sigma)
			\doteq \mathrm{Tr}\big(\hat\Sigma(\hat\Sigma+\lambda I)^{-1}\big)
			= \sum_j \frac{\hat\mu_j}{\hat\mu_j+\lambda}.
			\]
			
			Directions with $\hat\mu_j\gg\lambda$ contribute $\approx 1$ (active DoF);
			directions with $\hat\mu_j\ll\lambda$ contribute $\approx 0$ (frozen).
			
			\vspace{0.4em}
			\coloredbox[bgcolor=LightGray, framecolor=BerkeleyBlue, roundedcorners=4]{
				\centering
				\textbf{\large Spectral Risk Certificate}
				\vspace{0.3em}
				
				\[
				\boxed{
					\widehat{\mathcal{R}}_\lambda
					\doteq
					\underbrace{\lambda B^2}_{\text{worst-case bias}}
					+
					\underbrace{\frac{\sigma^2}{N} d_\lambda(\hat\Sigma)}_{\text{variance from geometry}}
				}
				\]
				
				\smallskip
				\textit{Computable from $X$ alone (unlabeled geometry).}
			}
		}
		
		% === SPECTRAL / DOUBLE-DESCENT INTUITION ===
		\block{Spectral Intuition}{
			\small
			Let $\alpha = p/N$ and consider small $\lambda$:
			\[
			\frac{d_\lambda(\hat\Sigma)}{N} \approx \frac{\min(p,N)}{N} = \min(\alpha,1).
			\]
			\begin{itemize}[leftmargin=*, itemsep=0.25em]
				\item $\alpha<1$: $d_\lambda/N \uparrow \alpha$ as $\lambda\to 0$.
				\item $\alpha>1$: $d_\lambda/N \uparrow 1$ and \textbf{saturates} (only $N$ samples).
				\item $d_\lambda/N \approx 1$: effective DoF $\approx$ sample size — the interpolation threshold.
			\end{itemize}
			
			Classical formulas suggest
			\[
			\text{Var} \sim \frac{d_\lambda/N}{1 - d_\lambda/N},
			\]
			which blows up at $d_\lambda/N=1$.
			
			\vspace{0.2em}
			\textbf{Prediction:} In $(d_\lambda/N,\;\text{risk})$ coordinates:
			\begin{itemize}[leftmargin=*, itemsep=0.25em]
				\item Nearly linear scaling of risk with $d_\lambda/N$ away from 1.
				\item A vertical ``spike'' near $d_\lambda/N=1$ (double-descent peak).
			\end{itemize}
		}
		
		% ---------------- RIGHT COLUMN ----------------
		\column{0.5}
		
		% === MAIN THEORETICAL RESULTS ===
		\block{Bias--Variance Decomposition \& Main Theorem}{
			
			\textbf{Excess empirical prediction error}
			\[
			\mathcal{E}_{\mathrm{emp}}(\theta;X)
			= \tfrac1N \|X(\theta-\theta^\star)\|_2^2
			= (\theta-\theta^\star)^\top \hat\Sigma (\theta-\theta^\star).
			\]
			
			For ridge $\hat\theta_\lambda$, let $\Delta_\lambda = \hat\theta_\lambda - \theta^\star$.
			Conditioned on $X$,
			\[
			\mathbb{E}[\mathcal{E}_{\mathrm{emp}}(\hat\theta_\lambda;X)\mid X]
			=
			\mathrm{Bias}_\lambda^2(X)
			+
			\mathrm{Var}_\lambda(X).
			\]
			
			\textbf{Theorem 1 (Fixed-design spectral risk bound).}
			
			Assume $\|\theta^\star\|_2\le B$ and
			$\mathbb{E}[\varepsilon\varepsilon^\top\mid X]\preceq \sigma^2 I_N$.
			Then for all $\lambda>0$,
			\[
			\mathbb{E}[\mathcal{E}_{\mathrm{emp}}(\hat\theta_\lambda;X)\mid X]
			\le
			\lambda B^2 + \frac{\sigma^2}{N} d_\lambda(\hat\Sigma).
			\]
			
			\textbf{Proof sketch:}
			\begin{itemize}[leftmargin=*, itemsep=0.2em]
				\item Express $\Delta_\lambda$ in eigenbasis of $\hat\Sigma$.
				\item Show
				$\mathrm{Bias}_\lambda^2(X)
				= \lambda^2 \theta^{\star\top}(\hat\Sigma+\lambda I)^{-1}
				\hat\Sigma(\hat\Sigma+\lambda I)^{-1}\theta^\star
				\le \lambda B^2$.
				\item Show
				$\mathrm{Var}_\lambda(X)
				\le (\sigma^2/N)\mathrm{Tr}[\hat\Sigma^2(\hat\Sigma+\lambda I)^{-2}]
				\le (\sigma^2/N)d_\lambda(\hat\Sigma)$.
			\end{itemize}
			
			\textbf{Interpretation.} Conservative: if $\widehat{\mathcal{R}}_\lambda$ is small, then risk is small even in the worst orientation of $\theta^\star$. When the variance term dominates, risk $\approx (\sigma^2/N)d_\lambda$,
				giving the linear trend in the collapse plot.
		}
		
		% === CONCENTRATION / CERTIFICATE STABILITY ===
		\block{Certificate Stability in Random Design}{
			
			Assume sub-Gaussian features with parameter $\kappa$.
			
			\textbf{Lemma (Lipschitz stability).}
			If $\|\hat\Sigma-\Sigma\|_{\mathrm{op}} \le \delta$, then
			\[
			|d_\lambda(\hat\Sigma) - d_\lambda(\Sigma)|
			\le \frac{\delta}{\lambda}\,\mathrm{rank}(\Sigma+\hat\Sigma).
			\]
			
			\textbf{Lemma (Covariance concentration).}
			w.p.\ $\ge 1-\eta$,
			\[
			\|\hat\Sigma-\Sigma\|_{\mathrm{op}}
			\le
			C_\kappa\left(
			\sqrt{\frac{\mathrm{Tr}(\Sigma)}{N}}
			+
			\sqrt{\frac{\log(1/\eta)}{N}}
			\right).
			\]
			
			\textbf{Corollary (Effective-dimension concentration).}
			Combining the two,
			\[
			d_\lambda(\hat\Sigma) \approx d_\lambda(\Sigma)
			\quad\text{with high probability.}
			\]
			
			\textbf{Spectral certificate.}
			The empirical quantity
			\[
			\widehat{\mathcal{R}}_\lambda
			= \lambda B^2 + \frac{\sigma^2}{N} d_\lambda(\hat\Sigma)
			\]
			concentrates around the population bound
			$\mathcal{R}^{\mathrm{pop}}_\lambda
			= \lambda B^2 + \frac{\sigma^2}{N} d_\lambda(\Sigma)$.
			
			\begin{itemize}[leftmargin=*, itemsep=0.2em]
				\item Works in $p\gg N$ (no assumption on $\lambda_{\min}(\hat\Sigma)$).
				\item If the certificate is small, both empirical and population risks are small: and we have 
				a scalar certificate of benignness.
			\end{itemize}
		}
		
	\end{columns}
	
	% ==============================================================================
	% MIDDLE SECTION: EMPIRICAL RESULTS
	% ==============================================================================
	
	\block{Empirical Validation: Geometry Predicts Benign Interpolation}{
		\small
		\begin{minipage}[t]{0.32\linewidth}
			\centering
			\textbf{1. Risk vs Effective Dimension}
			\vspace{0.2em}
			
			\includegraphics[width=0.95\linewidth]{risk_collapse.png}
			
			\vspace{0.2em}
			Excess risk vs.\ $d_\lambda(\Sigma)/N$ across aspect ratios
			$\alpha=p/N$ and regularization $\lambda$.
			Away from $d_\lambda/N\approx 1$, all points lie on an
			almost linear trend predicted by $(\sigma^2/N)d_\lambda$.
			The tall column at $d_\lambda/N\approx 1$ is the predicted
			interpolation ``singularity''.
		\end{minipage}\hfill
		\begin{minipage}[t]{0.32\linewidth}
			\centering
			\textbf{2. ROC: Benign vs Non-Benign}
			\vspace{0.2em}
			
			\includegraphics[width=0.95\linewidth]{roc.png}
			
			\vspace{0.2em}
			We label a model as benign if its test excess risk
			is below $0.05$.
			The spectral certificate $\widehat{\mathcal{R}}_\lambda$
			(blue/orange, AUC $\approx 0.90$) strongly outperforms:
			\begin{itemize}[leftmargin=*, itemsep=0.15em]
				\item effective dimension alone $d_\lambda/N$ (AUC $\approx 0.48$)
				\item classical $1/N$ scaling (AUC $\approx 0.25$)
				\item total variance $\mathrm{Tr}(\Sigma)$ (AUC $\approx 0.62$).
			\end{itemize}
		\end{minipage}\hfill
		\begin{minipage}[t]{0.32\linewidth}
			\centering
			\textbf{3. Certificate vs Actual Risk}
			\vspace{0.2em}
			
			\includegraphics[width=0.95\linewidth]{scatter.png}
			
			\vspace{0.2em}
			Each point is one trained model (various $\alpha$ and $\lambda$).
			X-axis: $\widehat{\mathcal{R}}_\lambda$ (unlabeled geometry).
			Y-axis: test excess risk.
			Most points lie between $y=x$ (black) and $y=5x$ (red):
			the certificate upper-bounds risk within a small constant
			factor over three orders of magnitude. $r^2 
			\approx .83$.
		\end{minipage}
	}
	
	% ==============================================================================
	% BOTTOM SECTION: SUMMARY & OUTLOOK
	% ==============================================================================
	
	\block{Summary, Contributions, and Outlook}{
		\small
		\begin{minipage}[t]{0.56\linewidth}
			\textbf{Main takeaways.}
			\begin{itemize}[leftmargin=*, itemsep=0.3em]
				\item A single scalar
				\[
				\widehat{\mathcal{R}}_\lambda
				= \lambda B^2 + \frac{\sigma^2}{N} d_\lambda(\hat\Sigma)
				\]
				computed from unlabeled features serves as a
				spectral certificate of benign generalization.
				\item Risk curves across widths and regularization
				collapse when parameterized by effective degrees
				of freedom $d_\lambda/N$, with a universal spike at
				$d_\lambda/N\approx 1$.
				\item Width and sample size matter only through the
				spectrum of the learned representation.
			\end{itemize}
			
			\vspace{0.3em}
			\textbf{Contributions.}
			\begin{enumerate}[leftmargin=*, itemsep=0.25em]
				\item A finite-sample bias--variance bound for ridge
				depending only on $d_\lambda(\hat\Sigma)$, valid for $p\gg N$.
				\item Concentration results showing that the empirical
				certificate tracks the ideal population bound
				without relying on $\lambda_{\min}(\hat\Sigma)$.
				\item Empirical evidence that the certificate almost
				linearly parameterizes risk and achieves AUC $\approx 0.9$
				for predicting benign vs non-benign interpolation.
			\end{enumerate}
		\end{minipage}\hfill
		\begin{minipage}[t]{0.4\linewidth}
			\textbf{Future directions.}
			\begin{itemize}[leftmargin=*, itemsep=0.25em]
				\item Apply spectral certificates to full deep nets
				(beyond last-layer linearization).
				\item Study robustness to label noise, distribution shift,
				and heavy-tailed features.
				\item Unsupervised choice of $\lambda$ from spectral geometry alone.
				\item Connections to NTK, kernel methods, and
				information-theoretic capacity measures.
			\end{itemize}
			
			\vspace{0.3em}
			\textbf{Stat mech view.}
			\begin{itemize}[leftmargin=*, itemsep=0.25em]
				\item $\sigma^2/N$ $\approx$ temperature / noise power.
				\item $d_\lambda$ counts active degrees of freedom.
				\item Benign interpolation arises when both
				energy ($\lambda B^2$) and thermal fluctuations
				$(\sigma^2/N)d_\lambda$ are small.
			\end{itemize}
		\end{minipage}
	}
	
\end{document}
